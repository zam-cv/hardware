\documentclass[11pt,letterpaper]{article}

% Packages basicos
\usepackage[utf8]{inputenc}
\usepackage[spanish]{babel}
\usepackage[margin=1in]{geometry}
\usepackage{fancyhdr}
\usepackage{graphicx}
\usepackage{amsmath}
\usepackage{enumitem}
\usepackage{array}
\usepackage{longtable}

% Configuracion de pagina
\pagestyle{fancy}
\fancyhf{}
\fancyhead[L]{Bitacora Practica 3}
\fancyhead[R]{Carlos Zamudio - A01799283}
\fancyfoot[C]{\thepage}
\renewcommand{\headrulewidth}{0.4pt}
\renewcommand{\footrulewidth}{0.4pt}

% Configuracion de lista
\setlist[itemize]{leftmargin=15pt, itemsep=2pt}
\setlist[enumerate]{leftmargin=15pt, itemsep=2pt}

\begin{document}

\begin{center}
    {\Large \textbf{BITACORA PRACTICA No. 3}}\\[0.3cm]
    {\large Dispositivos embebidos con sensores e Interfaces moviles}\\[0.5cm]
    \textbf{Nombre:} Carlos Alberto Zamudio Velazquez\\
    \textbf{Matricula:} A01799283\\
    \textbf{Modulo:} Integracion de hardware para ciencia de datos\\
    \textbf{Profesor:} Dr. David Higuera Rosales\\
    \textbf{Periodo:} 8 de septiembre - 18 de septiembre, 2025\\[0.5cm]
\end{center}

\section{OBJETIVOS DE LA PRACTICA}
\begin{itemize}
    \item Disenar soluciones de dispositivos embebidos con sensores
    \item Implementar hardware requerido segun especificaciones
    \item Obtener datos del puerto serie y almacenarlos
    \item Disenar interfaces basicas de interaccion para moviles
    \item Integrar Sistema de Monitoreo Ambiental Completo con ML y App Movil
\end{itemize}

\section{MATERIALES UTILIZADOS}
\begin{itemize}
    \item ESP32-C3 DevKit
    \item Sensor DHT22 (temperatura y humedad)
    \item LDR (sensor de luz) con resistencia 10k$\Omega$
    \item LEDs indicadores (azul, verde, rojo) con resistencias 220$\Omega$
    \item Protoboard y cables jumper
    \item Resistencia 10k$\Omega$ pull-up para DHT22
\end{itemize}

\section{BITACORA DIARIA}

\subsection{Domingo 8 de Septiembre, 2025}
\textbf{Analisis y Planificacion del Proyecto}

Se realizo el analisis inicial de requerimientos y la planificacion del sistema:
\begin{itemize}
    \item Analisis de requerimientos de la practica
    \item Investigacion sobre ESP32-C3 y sus capacidades
    \item Planificacion de arquitectura del sistema completo
    \item Definicion de tecnologias: ESP32-C3 + FastAPI + Flutter + PostgreSQL
\end{itemize}

\textbf{Decisiones tecnicas del equipo:}
\begin{itemize}
    \item Seleccionar ESP32-C3 por capacidades WiFi nativas
    \item Usar Flutter en lugar de React Native por mejor rendimiento en graficos
    \item PostgreSQL en lugar de CSV para almacenamiento robusto y consultas complejas
    \item FastAPI para backend con capacidades ML integradas
\end{itemize}

\subsection{Lunes 9 de Septiembre, 2025}
\textbf{Configuracion de Hardware}

Se realizaron las conexiones del hardware segun las especificaciones del ESP32-C3:

\textbf{Conexiones implementadas:}
\begin{itemize}
    \item DHT22: VCC $\rightarrow$ 3.3V, DATA $\rightarrow$ GPIO\_3, GND $\rightarrow$ GND
    \item Resistencia pull-up 10k$\Omega$ entre DATA y VCC del DHT22
    \item LDR: Un extremo $\rightarrow$ 3.3V, otro extremo $\rightarrow$ ADC\_CHANNEL\_4
    \item Resistencia 10k$\Omega$ desde ADC\_CHANNEL\_4 a GND (divisor de voltaje)
    \item LEDs: Azul (GPIO\_5), Verde (GPIO\_6), Rojo (GPIO\_7) con resistencias 220$\Omega$
\end{itemize}

\textbf{Pruebas realizadas por el equipo:}
\begin{itemize}
    \item Verificacion de lecturas DHT22: Precision $\pm$0.5°C, $\pm$3\% humedad
    \item Calibracion LDR: Rango 0-4095 mapeado a condiciones de luz
    \item Pruebas de conectividad WiFi exitosas
\end{itemize}

\textbf{Formulas utilizadas para calibracion:}
\begin{itemize}
    \item Voltaje LDR: $V_{LDR} = \frac{ADC_{value} \times 3.3V}{4095}$
    \item Resistencia LDR: $R_{LDR} = \frac{V_{LDR} \times 10k\Omega}{3.3V - V_{LDR}}$
\end{itemize}

\subsection{Martes 10 de Septiembre, 2025}
\textbf{Desarrollo del Firmware ESP32-C3}

Se implemento el sistema embebido utilizando ESP-IDF framework:

\textbf{Caracteristicas implementadas:}
\begin{itemize}
    \item Configuracion de tareas FreeRTOS para lectura no bloqueante de sensores
    \item Protocolo HTTP POST con autenticacion mediante header "protected"
    \item Sistema de reconexion WiFi automatica con event handlers
    \item Logica de indicadores LED basada en umbrales de temperatura
\end{itemize}

\textbf{Mi contribucion principal en el microcontrolador:}
\begin{itemize}
    \item Implemente la logica de envio de datos JSON al servidor FastAPI
    \item Configure el sistema de LEDs indicadores con umbrales personalizables
    \item Ajuste los parametros de muestreo de sensores y timing
    \item Optimice el formato de datos para compatibilidad con el backend
\end{itemize}

\textbf{Estructura de datos implementada:}
\begin{verbatim}
{
  "source": "ESP32",
  "sensor": "temperature|humidity|light",
  "value": <float_value>
}
\end{verbatim}

\textbf{Logica de LEDs programada:}
\begin{itemize}
    \item LED Azul: Temperatura < 20°C (frio)
    \item LED Verde: Temperatura entre 20°C y 30°C (normal)
    \item LED Rojo: Temperatura > 30°C (caliente)
\end{itemize}

\subsection{Miercoles 11 de Septiembre, 2025}
\textbf{Desarrollo del Backend FastAPI}

El equipo implemento el servidor backend con las siguientes caracteristicas:

\textbf{Arquitectura del servidor:}
\begin{itemize}
    \item Middleware de autenticacion con header "protected" personalizado
    \item Base de datos PostgreSQL con indices optimizados para time-series
    \item Modelos Pydantic para validacion de datos de entrada y respuesta
    \item Sistema de logging estructurado con diferentes niveles
\end{itemize}

\textbf{Esquema de base de datos implementado:}
\begin{verbatim}
CREATE TABLE metrics (
    id SERIAL PRIMARY KEY,
    timestamp TIMESTAMPTZ DEFAULT NOW(),
    source VARCHAR(100),
    sensor VARCHAR(50),
    value FLOAT
);

-- Indices para optimizar consultas temporales
CREATE INDEX idx_sensor_timestamp ON metrics(sensor, timestamp);
CREATE INDEX idx_timestamp_sensor ON metrics(timestamp, sensor);
\end{verbatim}

\textbf{Endpoints de API implementados:}
\begin{itemize}
    \item POST /metric - Recepcion de datos de sensores desde ESP32
    \item GET /metrics/history - Consulta historica con filtros temporales
    \item WebSocket /ws-metrics - Stream de datos en tiempo real
    \item GET /predict/\{sensor\_type\} - Predicciones ML multi-horizonte
\end{itemize}

\subsection{Jueves 12 de Septiembre, 2025}
\textbf{Implementacion Machine Learning}

Se desarrollo el sistema de predicciones usando LightGBM:

\textbf{Sistema de predicciones implementado:}
\begin{itemize}
    \item Entrenamiento automatico cada 6 horas con minimo 10 puntos de datos
    \item Caracteristicas temporales: medias moviles, tendencias, estacionalidad
    \item Predicciones multi-horizonte: 15min, 1h, 6h, 24h
    \item Calculo de intervalos de confianza basados en varianza historica
\end{itemize}

\textbf{Feature Engineering aplicado:}
\begin{verbatim}
def create_features(data):
    features = [
        data['value'].rolling(5).mean().iloc[-1],    # Media 5 puntos
        data['value'].rolling(10).mean().iloc[-1],   # Media 10 puntos
        (data['value'].iloc[-1] - data['value'].iloc[-5]) / 5,  # Tendencia
        datetime.now().hour,                         # Hora del dia
        datetime.now().weekday()                     # Dia de la semana
    ]
    return np.array(features)
\end{verbatim}

\textbf{Metricas de rendimiento obtenidas:}
\begin{itemize}
    \item MAE temperatura: 1.1°C para predicciones 1h
    \item MAE humedad: 4.2\% para predicciones 1h
    \item Tiempo de inferencia: <50ms promedio
    \item Precision 15min: 96\% dentro de $\pm$1°C
\end{itemize}

\subsection{Viernes 13 de Septiembre, 2025}
\textbf{Desarrollo Aplicacion Flutter}

Se desarrollo la aplicacion movil con arquitectura modular:

\textbf{Arquitectura implementada:}
\begin{itemize}
    \item Patron MVC con servicios separados para API y WebSocket
    \item Gestion de estado reactiva con StreamBuilder para datos en tiempo real
    \item Biblioteca fl\_chart para visualizaciones interactivas
    \item Sistema de notificaciones locales con permission\_handler
\end{itemize}

\textbf{Servicios implementados en la app:}
\begin{itemize}
    \item ApiService - Cliente HTTP con autenticacion automatica
    \item WebSocketService - Conexion persistente con reconexion automatica
    \item LocalNotificationService - Alertas push locales
    \item LocalAlarmService - Sistema de alarmas configurables
\end{itemize}

\textbf{Mi contribucion principal en la aplicacion:}
\begin{itemize}
    \item Implemente el sistema de graficos interactivos con fl\_chart
    \item Configure las notificaciones push y alarmas locales
    \item Desarrolle la logica de reconexion automatica del WebSocket
    \item Diseñe la interfaz de usuario del dashboard principal
    \item Integre el sistema de predicciones ML en la UI
\end{itemize}

\textbf{Pantallas principales desarrolladas:}
\begin{itemize}
    \item Dashboard con cards de sensores actualizadas en tiempo real
    \item Graficos de linea temporal con zoom y pan interactivo
    \item Pantalla de predicciones ML con intervalos de confianza
    \item Configuracion de alarmas personalizables por sensor
\end{itemize}

\subsection{Sabado 14 de Septiembre, 2025}
\textbf{Integracion WebSocket Tiempo Real}

Se implemento el sistema de comunicacion bidireccional:

\textbf{Flujo de datos en tiempo real:}
\begin{enumerate}
    \item ESP32-C3 envia datos via HTTP POST cada 5 segundos
    \item FastAPI almacena en PostgreSQL y encola para distribución WebSocket
    \item Flutter recibe actualizaciones y actualiza UI reactivamente
    \item Sistema evalua umbrales y dispara notificaciones automaticamente
\end{enumerate}

\textbf{Optimizaciones implementadas:}
\begin{itemize}
    \item Cola asincrona en FastAPI para distribucion eficiente
    \item Reconexion automatica en Flutter con backoff exponencial
    \item Buffer local para manejo de desconexiones temporales
    \item Compresion de datos para optimizar ancho de banda
\end{itemize}

\subsection{Domingo 15 de Septiembre, 2025}
\textbf{Sistema de Alertas y Notificaciones}

Se completo el sistema de alertas multi-modal:

\textbf{Alertas hardware (ESP32-C3):}
\begin{itemize}
    \item Logica de LEDs implementada segun umbrales de temperatura
    \item Frecuencia de parpadeo proporcional a desviacion del rango normal
    \item Estados visuales: Frio (azul), Normal (verde), Caliente (rojo)
\end{itemize}

\textbf{Notificaciones moviles implementadas:}
\begin{itemize}
    \item Push notifications para valores criticos de sensores
    \item Alarmas programables con intervalos personalizables
    \item Persistencia en SharedPreferences para configuracion de usuario
    \item Multiples umbrales configurables por tipo de sensor
\end{itemize}

\subsection{Lunes 16 de Septiembre, 2025}
\textbf{Pruebas y Optimizacion del Sistema}

Se realizaron pruebas exhaustivas del sistema completo:

\textbf{Pruebas de sistema ejecutadas:}
\begin{itemize}
    \item Conectividad: 24h continuas sin desconexiones criticas
    \item Carga: 50 clientes WebSocket simultaneos sin degradacion
    \item Precision ML: Validacion cruzada con 500+ puntos historicos
    \item Autonomia: Aplicacion movil funcional durante 6h+ de uso continuo
\end{itemize}

\textbf{Optimizaciones aplicadas:}
\begin{itemize}
    \item Indices compuestos en BD para consultas temporales (90\% mejora)
    \item Lazy loading en graficos historicos de Flutter
    \item Pool de conexiones PostgreSQL para alta concurrencia
\end{itemize}

\textbf{Resultados de rendimiento obtenidos:}
\begin{itemize}
    \item Latencia API: 45ms promedio para endpoints basicos
    \item Throughput: 800+ requests/segundo en condiciones normales
    \item Tiempo real completo: 180ms desde sensor hasta aplicacion
    \item Disponibilidad del sistema: 99.8\% durante pruebas de 48h
\end{itemize}

\subsection{Martes 17 de Septiembre, 2025}
\textbf{Documentacion}

\textbf{Validacion de requerimientos completada:}
\begin{itemize}
    \item [✓] Datos ambientales almacenados (PostgreSQL supera CSV)
    \item [✓] API REST funcional con capacidades ML avanzadas
    \item [✓] Aplicacion movil Flutter completamente funcional
    \item [✓] Sistema de alertas multi-modal (LED + notificaciones)
    \item [✓] Analisis predictivo con metricas de confianza
\end{itemize}

\section{ARQUITECTURA FINAL DEL SISTEMA}

\begin{center}
\begin{tabular}{|c|c|c|}
\hline
\textbf{ESP32-C3} & \textbf{FastAPI Server} & \textbf{Flutter App} \\
\hline
DHT22 + LDR & PostgreSQL & Dashboard \\
LEDs indicadores & Machine Learning & Graficos tiempo real \\
WiFi + HTTP & WebSocket & Notificaciones \\
\hline
\end{tabular}
\end{center}

\textbf{Flujo de datos implementado:}\\
Sensores $\rightarrow$ ESP32-C3 $\rightarrow$ HTTP POST $\rightarrow$ FastAPI $\rightarrow$ PostgreSQL\\
FastAPI $\rightarrow$ WebSocket $\rightarrow$ Flutter App $\rightarrow$ Usuario

\section{PROBLEMAS ENCONTRADOS Y SOLUCIONES}

\subsection{Desafios de Hardware}
\textbf{Problema:} Lecturas erraticas iniciales del DHT22\\
\textbf{Solucion aplicada:} Agregar resistencia pull-up 10k$\Omega$ y delay de inicializacion de 2 segundos

\textbf{Problema:} Ruido en lecturas ADC del LDR\\
\textbf{Solucion aplicada:} Promedio de 10 lecturas consecutivas con filtro de mediana

\textbf{Problema:} Desconexiones WiFi esporadicas\\
\textbf{Solucion aplicada:} Event handlers para reconexion automatica con retry exponencial

\subsection{Desafios de Software}
\textbf{Problema:} Desconexiones WebSocket frecuentes en la aplicacion\\
\textbf{Solucion aplicada:} Implementar reconexion automatica con backoff exponencial

\textbf{Problema:} Latencia alta en consultas historicas de gran volumen\\
\textbf{Solucion aplicada:} Indices compuestos en BD y paginacion de resultados

\textbf{Problema:} Lag en graficos de Flutter con muchos puntos\\
\textbf{Solucion aplicada:} Lazy loading y decimacion inteligente de datos

\section{CONCLUSIONES}

\subsection{Objetivos Alcanzados}
\begin{itemize}
    \item Sistema completo funcional integrando hardware, backend y aplicacion movil
    \item Reemplazo exitoso de CSV con base de datos PostgreSQL
    \item Implementacion de Machine Learning predictivo
    \item Desarrollo de aplicacion Flutter
    \item Sistema de alertas multi-modal (LED + notificaciones) funcional
    \item Comunicacion tiempo real WebSocket estable y confiable
\end{itemize}

\subsection{Aprendizajes Tecnicos Principales}
\begin{itemize}
    \item Dominio de ESP32-C3 y sensores ambientales de precision
    \item Desarrollo de APIs REST asincronas con FastAPI
    \item Implementacion de modelos ML para prediccion de series temporales
    \item Desarrollo de aplicaciones moviles nativas con Flutter
    \item Integracion completa de sistemas IoT complejos
    \item Optimizacion de bases de datos para aplicaciones time-series
\end{itemize}

\subsection{Contribuciones Personales Destacadas}
\textbf{En el microcontrolador ESP32-C3:}
\begin{itemize}
    \item Logica de envio de datos JSON estructurado
    \item Sistema de LEDs indicadores con umbrales configurables
    \item Optimizacion de parametros de muestreo
\end{itemize}

\textbf{En la aplicacion Flutter:}
\begin{itemize}
    \item Sistema de graficos interactivos con fl\_chart
    \item Logica de notificaciones push y alarmas
    \item Reconexion automatica del WebSocket
    \item Diseño de interfaz de usuario del dashboard
    \item Integracion de predicciones ML en la UI
\end{itemize}

\vspace{0.5cm}
\noindent Este proyecto demuestra la integracion exitosa de tecnologias embebidas, backend moderno y aplicaciones moviles para crear un sistema de monitoreo ambiental completo y escalable.

\end{document}